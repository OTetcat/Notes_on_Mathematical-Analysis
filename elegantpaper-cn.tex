%!TEX program = xelatex
% 完整编译方法 1 pdflatex -> bibtex -> pdflatex -> pdflatex
% 完整编译方法 2: xelatex -> bibtex -> xelatex -> xelatex
\documentclass[lang=cn,11pt]{elegantpaper}

\title{关于微积分的基本技巧}
\author{%
	%\href{https://ddswhu.me/}
	{せいひつ}}

% 不需要版本信息,直接注释即可
%\version{0.07}
% 不需要时间信息的话,需要把 \today 删除。
\date{\today}

\newcommand{\tit}[1]{\textit{#1}}
\newcommand{\tbf}[1]{\textbf{#1}}
\newcommand{\trm}[1]{\textrm{#1}}
\newcommand{\cmt}[1]{}

% 如果想修改参考文献样式,请把这行注释掉
%\usepackage[authoryear]{gbt7714}  % 国标

\begin{document}

\maketitle

\begin{abstract}
\begin{center} 本文档类名为 \lstinline{}{elegantpaper},不是作者所作,也没有修改些什么 \end{center}
\noindent 这是一篇在基本计算上所完成的札记,内容大致包含一般会出现的各式极限、微分与积分的计算,也就是从各教材中(在主观意义下)提炼的重点知识,希望不会太长
\end{abstract}

\section{前言}
就像在教科书中所看到那样,我们在做运算时会遇到几类最基本的函数:\tit{有理函数,三角函数,幂级数},统称为\tbf{代数}与\tbf{初等函数}。尽管在做一些超越运算 \tit{- 如积分 -} 的时候,它们最终得到的结果也许是\tit{超越}的,但一般来说亦可以用统一的形式求其极限与微分/导数。
%\tit{基本原则:把所有函数化为形如有理函数的形式,即幂级数的比值。}

我们可以在此之上套用许多方法,代表者如\trm{Taylor}\tit{展开、三角反/%
%\footnote{符号 “ a / b ” 代表 “ a 与 ab ”,如\tit{反/三角函数}代表\tit{反三角与三角函数}}
代换、双曲反/代换},本文中会详细讨论它们的应用范围与主要用途。简述如下:(文档架构)
\begin{enumerate}
	\item 所需的引理与结论:注释、摘记,证明不常见或课本略去不证的命题/定理,给出思路
	\item 最一般的结果:代数函数的极限与导数,代数函数的分类
	\item 有理函数:部分分式分解、积分,三角变换初探
%	\item 复化:用复方法求解实积分的一般形式,三角变换的理论
	\item 代数函数:超越函数的出现,有理化,几类标准积分
\end{enumerate}

\section{绪论}
%每一个小节都是一个关键的定理,并且不提供例子与用途上的拓展
\subsection{有理化}
\begin{theorem}
	以下论述等价:令\tit{多项式}为\tbf{实系数幂级数},则
	\begin{enumerate}
		\item 所有有理函数都可以化为形如下述三者的线性组合:
		\[
			a_kx^k \; \text{,} \; \frac{A}{(x+a)^n} \; \text{,} \; \frac{Bx+C}{(x^2+px+q)^m}
		\]
		
		\item 所有(首一)多项式都可视为一次与二次式的乘积,即
		\[
			P(x) = \prod_i(x-x_i)^{k_i}\prod_j(x^2+p_jx+q_j)^{k_j} \; \text{,其中} k_i, k_j \; \text{是相关表达式的重根数}.
		\]
		
		\item 所有 $ \mathrm{deg}(P) = n $ 的多项式 $ P(x) $ 都有 $ n $ 个根(d'Alembert 定理),同时其复根的共轭亦为其根
	\end{enumerate}
\end{theorem}
\begin{proof}
由于 d'Alembert 定理,我们可以把任何一个多项式写为以下标准型:
\[
P(x) = a_0(x-x_1)(x-x_2)\dots(x-x_n) \; \text{,其中} \; \mathrm{deg}(P) = n.
\]
再由共轭复根定理,上述多项式可以改写为
\begin{align*}
	P(x) &= a_0\prod_{i=1}^{m_1}(x-x_i)^{k_i} \, \cdot \, \prod_{j=1}^{m_2}(x-(u+\mathrm{i}v)(x-(u-\mathrm{i}v)))^{k_j}\\
	&= a_0\prod_{i=1}^{m_1}(x-x_i)^{k_i} \, \cdot \, \prod_{j=1}^{m_2}(x^2-(2u)x+(u^2+v^2))^{k_j}\\
	&= a_0\prod_i(x-x_i)^{k_i}\prod_j(x^2+p_jx+q_j)^{k_j}
\end{align*}
\end{proof}

% 如果想修改参考文献样式(非国标),请把下行取消注释,并换成合适的样式(比如 unsrt,plain 样式)。
%\bibliographystyle{aer}
%\bibliography{wpref}

\end{document}
